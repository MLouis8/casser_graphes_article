\section{Methodology}
\label{sec:methodology}

\subsection*{Graph \& Model}
In this paper, we based our work on a Paris graph model. It is built with Open Street Map data with the Python package OSMnx. After a small preprocessing we have a graph of Paris with roads represented as edges and nodes as roads intersections. With the specifications in Figure~\ref{fig:characterization}. Since we used KaHIP for partitioning, we used an undirected and a parallel-edgeless version of the graph.
\begin{figure}[!hbt]
    \begin{center}
        \begin{tabular}{ | c c | }
        \hline
         \# of vertices & 40593 \\
        \hline 
         \# of edges & 46759 \\
        \hline
         avg. degree & 2.3 \\
        \hline
        \end{tabular}
        \caption{Paris network characterization}
        \label{fig:characterization}
    \end{center}
\end{figure}

\subsection*{Weight}
We then weighted the graph by the number of street lanes (a larger road is considered harder to disrupt). OSM Paris data begin quite heterogenous we used the following rules for a weight $w$ (taking the first rule where the parameter is given):
\begin{enumerate}
    \item $w =$ number of lanes
    \item $w =$ \textit{width} $/ 4$
    \item $w = 3$ if \textit{highway} $\in$ \{\textit{primary, secondary}\} \item $w = 2$ otherwise
\end{enumerate}
Keeping in mind that the number of lanes is given for 51\% of the edges, the width 3\% and the \textit{highway} for 100\% (it describes the type of road). We obtain the weight distribution of Figure~\ref{fig:lanes}.

\begin{figure}
    \centering
    \includegraphics[scale=0.5]{degree.pdf}
    \caption{Paris network degree distribution}
    \label{fig:degree}
\end{figure}

\begin{figure}
    \centering
    \includegraphics[scale=0.5]{lanes_distr.pdf}
    \caption{Paris network weight distribution}
    \label{fig:lanes}
\end{figure}

% \begin{center}
%     \includegraphics[scale=0.5]{valuation.pdf}
% \end{center}

\begin{figure}
    \begin{multicols}{2}
        \includegraphics[scale=0.8]{cuts.pdf}
        \begin{boxB}
            \begin{center}
                \emph{Legend:}\\
                \color{red} cut 190: 42 edges and 114 lanes\\
                \color{violet} cut 141: 53 edges and 130 lanes\\
                \color{orange} louvain: 81 edges and 184 lanes
            \end{center}
        \end{boxB}
    \end{multicols}
    \caption{Paris network with 3 different cuts and their characteristics}
    \label{fig:cuts}
\end{figure}
