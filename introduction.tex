\section{Introduction}
\label{sec:introduction}
The system of streets and intersections in a city, models a real-world transport network where traffic flow and movement can be analysed. As every network it's subject to disruptions, being natural failures or targeted attacks. In the study of protest, we could see a street blocking as an edge removal in the corresponding graph. This suggest adapting the model to the context, as a boulevard is harder to block than a small street.\\
\emph{Problématique}\\
We decided to explore a partitioning approach to create partition-based strategies in order to pertubates efficiently our network. In this paper KaHIP algorithm is used (\textbf{ref}) but the same methodoly could be reused with another partitioning algorithm.\\
KaHIP or Karlsruhe High Quality Partitioning is a family of graph partitioning programs. In our work we used KaFFPa (Karlsruhe Fast Flow Partitioner) a sequential Multi-level Graph Partitioning scheme, that computes high quality partitions for big graphs when an imbalance $i > 0$ is allowed. \emph{For small imbalances, we observed very small isolated components, which increases artificially the cost of the cut.} In this way we could face a NP hard problem with a fast and good approximation. And we will see later to what extent the edges selected by KaHIP are interresting for our problem.\\
Since its an approximation, when repeating the cut with different seeds, different cuts may appear. They often reproduce the same pattern, but very different cuts may occur. In order to understand the diversity of the outcomes produced by KaFFPa, we developped a cut clustering method. To face this set-clustering problem, we used the Chamfer distance together with the Louvain clustering algorithm on a distance graph established from a custom treshold. Thanks to this method we easily visualized the many possible partitions created by KaFFPa.\\
Similar outcomes could have been reached using clustering detection tools. But using a partitioning point of vue allowed us to manipulate this balancing perspective and removing in the same time any density constraint on the output (something less pertinent on a city graph, where degrees are quite homogenous \textbf{figure}).

state of the art strategies

To compare and analyse 
\begin{enumerate}
    \item \textbf{Largest Connected Component}
    Classic robustness metric, the bigger the its size is, the more the graph is considered robust.\\
    \textit{Complexity: $O(n)$}

    \item \textbf{Efficiency}
    The efficiency of a pair of nodes in a graph is the multiplicative inverse of the shortest path distance between the nodes. The average global efficiency of a graph is the average efficiency of all pairs of nodes. The better the graph is connected, the lesser the average distance is so bigger the efficiency is.\\
    \textit{Complexity: $O(n^2)$}

    \item \textbf{Effective Resistance}
    Views the resistance as the electrical resistance between two vertices using Kirchoff's circuit laws. The effective graph resistance is the sum of every vertex pair resistance. But it's computed using the sum of the inverse of non-zero Laplacian eigenvalues.
    $$R = n \sum_{i=2}^n\frac{1}{\mu_i}$$
    It measures the graph connectivity taking into account both the number of paths between node pairs and their length.\\
    \textit{Complexity: $O(n^3)$}
\end{enumerate}


    \textbf{Iterative eBC vs non iterative eBC}\\
    \vspace{1pt}\\
    The use of iterative eBC computations is expensive but necessary. A first order on the eBC shows zones of multiple edges with high eBC. And cutting the biggest eBC edge reduces the eBC in all the area, making useless other cuts in the same zone. For this reason we always recompute the eBC. For big attacks, an approximation is made with 1000 random samples.\\