\documentclass{article}

\usepackage{graphicx}
\graphicspath{{images/}}

\begin{document}

\section{Intro}

\section{Related Works}

\section{Choix modélisation}
\subsection{Graphe}
\begin{itemize}
    \item Osmnx donc données d'OSM ave granularité de 4m car moyenne des rues Parisiennes
    \item Limites du modèle
    \item paramètres de preprocessing pour repoductibilité ? (driveway, projection, buffer...)
    \item contrainte de KaHIP $\rightarrow$ graphe non orienté\\
    Donc toutes les coupes sont faites sur le graphe non orienté. Lorsque le graphe orienté est évoqué (notamment lors de l'analyse des composantes fortement connexes) on 'désoriente le graphe'
    
\end{itemize}

\subsection{Valuation}
Valuation par les lanes, osmnx et limites de osm et de la valuation.\\
\textbf{Différence entre non valué et valué pour la eBC ?}\\
ou\\
\textbf{Carte des coûts ?}\\
ou\\
\textbf{Distribution des coûts ?}

\section{Coupes}
\begin{itemize}
    \item Coupes nat. dans les villes
    \item KaHIP
    \begin{itemize}
        \item[-] coupe équilibrée, imbalance
        \item[-] composantes connexes $\rightarrow$ phénomènes étranges
        \item[-] \underline{clusters de coupes $\rightarrow$ fréquence et diversité des coupes}
        \item[-] efficacité de l'algo (pourquoi baser une stratégie dessus est intéressant)
    \end{itemize}
    \textbf{Coupe 190 (et 24 et 141 ?), avec nb d'arêtes enlevées et poid}
    \item \underline{Clustering de coupe:}
    \begin{itemize}
        \item[-] Méthode, distance de Chamfer, algo de clustering (Louvain et BIRCH modifié ou BallTree)
        \item[-] Résultats sur plusieurs valuations, diversité des coupes 
    \end{itemize}
    \item Communautés\\
    \textbf{"coupe" à la Louvain et caractéristiques}
    % \centering
    % \includegraphics[scale=0.5]{coupe_louvain.pdf}
\end{itemize}

\section{Robustesse}
\subsection{Métriques d'impact}
\begin{itemize}
    \item eBC et average eBC
    \item SCC et CC
    \item Efficacité
    \item Résistance effective
\end{itemize}

A chaque fois définition et intuition, compléxité pratique et théorique (y penser lorsqu'on base une stratégie dessus)

\subsection{Stratégies}

    Définition et premières analyses des stratégies\\
    \emph{Classiques}
    \begin{itemize}
        \item[-] eBC, itératif vs non itératif
        \item[-] degré, inefficace sur graphe de ville
        \item[-] random comme stratégie naïve 
    \end{itemize}
    \emph{Basées sur KaHIP}
    \begin{itemize}
        \item[-] fréquence, explication et dominance d'un style de coupe
        \item[-] coupes, imposition d'un ordre dans la coupe \textbf{comparaison bc vs rd}
    \end{itemize}
    \emph{Passage à l'échelle}
    \begin{itemize}
        \item[-] approximation de la eBC
        \item[-] itération de la fréquence, changement de imb, de nblock
    \end{itemize}
\subsection{Résultats}
Comparaisons des stratégies entre elles\\
\textbf{Graphiques des comparaisons par métrique}

\subsection{Autres villes}
\begin{itemize}
    \item Manhattan
    \item Shanghai
    \item Ville non occidentale ?
\end{itemize}
Comparaisons et conclusions inter-villes.
\end{document}
