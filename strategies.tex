\section{Strategies}
\label{sec:strategies}
To formalize the different attacks on a graph we introduce strategy. It's the strategies that will be later analysed and compared in the section \ref*{sec:results}.
\begin{definition}(strategy).
    Let a strategy on a graph $G = (V, E)$ be a pair $(E', \leq)$ with $E'\subset E$ and $\leq$ an order on the edges of $E'$. Applying a strategy $(E', \leq)$ resides in the repetition of:
    \begin{enumerate}
        \item let $e^* = \max_{e\in E'}{e}$
        \item remove $e^*$ and recompute the order
    \end{enumerate}
\end{definition}

Two types of strategies have been manipulated, the first part are state of the art strategies, using centrality measures to create an order on the entire graph. In this category lies also the random strategy used as a naive baseline. The second type of strategies are the one we introduced, based on a partitioning algorithm.
\subsection*{State of the art strategies}
\begin{definition}(eBC strategy).
    The \textbf{edge Betweenness Centrality} strategy is the pair $(E, \leBC)$ such that $eBC: E \to \R$ is the classic edge Betweenness Centrality function defined as:
    $$eBC(e) = \sum_{s, t \in V}\frac{\sigma(s, t | e)}{\sigma(s, t)} \quad e\in E$$
    with $\sigma(s, t)$ the number of shortest $s-t$ paths. So $$\forall (u, v) \in E^2, u \leBC v \text{ when } eBC(u) \leq eBC(v)$$
\end{definition}
\textit{eBC} for the entire graph is always recomputed, it's expensive but necessary. If it was only computed once, we'll have small zones of edges with high eBC. And cutting once in the zone reduces the eBC drastically in all the area, making useless other cuts in the same zone. For big attacks (over 50 edges), an approximation is made with 1000 random samples.

\begin{definition}(deg strategy).
    The \textbf{Maximum Degree} strategy is the pair $(E, \ldeg)$ such that $deg: E \to \N$ is the edge degree fonction defined as: $deg((s, t)) = degree(s) \cdot degree(t),\quad (s, t)\in E$. So
    $$\forall (u, v) \in E^2, u \ldeg v \text{ when } deg(u) \leq deg(v)$$
    with $deg: V \to \N$ the common degree function.
\end{definition}

\begin{definition}(rd strategy).
    The \textbf{Random} strategy is the pair $(E, \lrd)$ such that $rd: E \to \N$ is a random injection. So
    $$\forall (u, v) \in E^2, u \lrd v \text{ when } rd(u) \leq rd(v)$$
\end{definition}
    
\subsection*{KaHIP based strategies}

\begin{definition}(freq strategy).
    The \textbf{frequency} strategy is the pair $(E, \lfreq)$. Let $N \in \N$ the number of cuts, $i \in  [0, 1]$ the imbalance parameter and $k \in \N$ the number of blocks the cut produces, we define $freq_{N, i, k}: E \to [0, 1]$ the function maping each edge $e \in E$ to its occurence divided by $N$. So
    $$\forall (u, v) \in E^2, u \lfreq v \text{ when } freq_{i, k}(u) \leq freq_{i, k}(v)$$
\end{definition}
For smaller attacks (untill 50 edges) $N = 1000$ allows a good overview of the cuts possibilities. For bigger attacks $N = 10$ since it have been tested that bigger values won't utterly change the results.

\begin{definition}(cut strategy).
    The \textbf{cut} strategy is the pair $(E_n, \leq_o)$. With $n$ a cut id and $E_n \subset E$ the set of edges corresponding to the cut. Let $o$ an order on the edges such that $o \in \{eBC, deg, rd\}$.
\end{definition}
Two particular cuts are considered in this paper, the \emph{cut141} a vertical cut and \emph{cut190} the best cut among a set of 1000 cuts (they're displayed in Figure~\ref{fig:cuts}). Both are cuts made with $i = 0.05$ and $k = 2$ (imbalance and number of blocks).