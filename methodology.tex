\section{Methodology}
\label{sec:methodology}

\subsection*{Graph \& Model}
In this paper, we based our work on a Paris graph model. It is built with Open Street Map data with the Python package OSMnx. \textit{We first download OSM data, then project them with epsg:2154, consolidate intersections with a tolerance of 4m and use  350m buffer around Paris.} Resulting from this preprocessing we have a graph of Paris with roads represented as edges and nodes as roads intersections. Its specifications can be found in \textbf{figure}. Since we used KaHIP for partitioning, we used an undirected and a parallel-edgeless version of the graph.

\subsection*{Weight}
We then weighted the graph by the number of street lanes (a larger road is considered harder to disrupt). OSM Paris data begin quite heterogenous we used the following rules for a weight $w$:
\begin{enumerate}
    \item $w =$ number of lanes
    \item $w =$ \textit{width} $/ 4$
    \item $w = 3$ if \textit{highway} $\in$ \{\textit{primary, secondary}\} \item $w = 2$ otherwise
\end{enumerate}
Knowing that the number of lanes is given for 51\% of the edges, the width 3\% and the \textit{highway} for 100\% (it describes the type of road).\\

\includegraphics[scale=0.5]{degree.pdf}
\includegraphics[scale=0.5]{lanes_distr.pdf}
% \begin{center}
%     \includegraphics[scale=0.5]{valuation.pdf}
% \end{center}


\begin{multicols}{2}
    \includegraphics[scale=0.8]{cuts.pdf}
    \begin{boxB}
        \begin{center}
            \emph{Legend:}\\
            \color{red} cut 190: 42 edges and 114 lanes\\
            \color{violet} cut 141: 53 edges and 130 lanes\\
            \color{orange} louvain: 81 edges and 184 lanes
        \end{center}
    \end{boxB}
\end{multicols}