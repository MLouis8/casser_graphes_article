\section{Introduction}
\label{sec:introduction}
The system of streets and intersections in a city, models a real-world transport network where traffic flow and movement can be analysed. As in other networks, it is subject to disruptions such as natural failures or targeted attacks. In the study of protests, we could see a roadblocks as an attack, where a blocked street is modeled as an edge removal in the corresponding graph. This implies adapting the model to the context (for example a boulevard is harder to block than a small street). In order to select edges important to the fonctionality of the network, targeted attacks often rely on centrality measures. But in this paper, we decided to explore a partitioning algorithm to create partition-based attacks. How efficient can partioning-based attacks be against a street network ?\\
In this paper we used KaHIP algorithm but the same methodoly could be re-used with another partitioning algorithm. KaHIP or Karlsruhe High Quality Partitioning \cite{sanders2012think} is a family of graph partitioning programs. In our work we used KaFFPa (Karlsruhe Fast Flow Partitioner) a sequential Multi-level Graph Partitioning scheme, that computes high quality partitions for big graphs when an imbalance $i > 0$ is allowed. Meaning that the output partitions aren't perfectly balanced, they are so by a factor $i$. Since its an approximation, we face a NP hard problem with a fast and good approximation. And when repeating the cut with different seeds, different cuts may appear. They often reproduce the same pattern, but in rare cases very different cuts appear. In order to understand the diversity of the outcomes produced by KaFFPa, we developped a cut clustering method. To face this set-clustering problem, we used the Chamfer distance together with the Louvain clustering algorithm on a distance graph established from a custom treshold. Thanks to this method we easily visualized the many possible partitions created by KaFFPa, with the objective to extract cuts with interresting properties.\\
Similar partitioning outcomes could have been reached using clustering detection tools (see in Figure~\ref{fig:cuts} the Louvain "cut"). But using a partitioning point of vue allowed us to manipulate this balancing perspective and removing in the same time any density constraint on the output (something less pertinent on a city graph, where degrees are quite homogenous as seen in Figure~\ref{fig:characterization} and Figure~\ref{fig:degree}).\\
Comparing and evaluating different attacks requires defining a set of robustness metrics, and we will use the following three: largest connected component size (LCC), efficiency and efffective resistance. LCC is a classic robustness metric, the bigger it is, the more parts of the network are connected. So the more the graph is considered robust. \textit{LCC complexity: $O(n)$}. The efficiency of a pair of nodes in a graph is the multiplicative inverse of the shortest path distance between the nodes. The average global efficiency (efficiency) of a graph is the average efficiency of all pairs of nodes. The better the graph is connected, the lesser the average distance is, so bigger the efficiency is. \textit{Efficiency complexity: $O(n^2)$}. The effective resistance views the resistance as the electrical resistance between two vertices using Kirchoff's circuit laws. The effective graph resistance is the sum of every vertex pair resistance. But it's computed using the sum of the inverse of non-zero Laplacian eigenvalues:
    $$R = n \sum_{i=2}^n\frac{1}{\mu_i}$$
It measures the graph connectivity taking into account both the number of paths between node pairs and their length. So a lower effective resistance implies more alternative pathways so a higher robustness. \textit{Effective resistance complexity: $O(n^3)$}.